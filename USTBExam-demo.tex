% !TEX encode = UTF-8
% !TEX program = xelatex
\documentclass{USTBExam}

\answerfalse %不显示答案

\newcommand{\cov}{\operatorname{cov}}

\begin{document}

\renewcommand{\niandu}{2017--2018}
\renewcommand{\xueqi}{二}
\renewcommand{\kecheng}{微积分AII}
\renewcommand{\shijuan}{A}
\setcounter{TotalPart}{4}

% 生成试卷表头
\makehead

\makepart{填空题}{共~8~小题,每小题~2~分,共~16~分}

\begin{problem}
已知二阶行列式 $\text{$\left|\begin{array}{cc}
  1 & 2\\
  - 3 & x
\end{array}\right|$=0}$,则 $x=$ \fillin{$-6$}。
\end{problem}

% \vfill

\begin{problem}
五阶行列式的一共有 \fillin{$120$} 项。
\end{problem}

% \vfill

\begin{problem}
向量组 $\alpha_1=(1,1,0), \alpha_2=(0,1,1), \alpha_3=(1,0,1)$,
则将向量 $\beta=(4, 5, 3)$ 表示为 $\alpha_1, \alpha_2, \alpha_3$
的线性组合为 $\beta=$ \fillin{$3\alpha_1+2\alpha_2+\alpha_3$}。
\end{problem}

% \vfill

\begin{problem}
已知$P(A)=0.3$, $P(B|A)=0.4$, $P(B|\bar{A})=0.5$, 则$P(B)=$ \fillin{$0.47$}。
\end{problem}

% \vfill

\begin{problem}
已知连续型$\xi$的密度函数为$\varphi(x)=\left\{
\begin{array}{ll}
  k \cos x, & - \frac{\pi}{2} < x < \frac{\pi}{2}\\
  0, & \text{其它}
\end{array}\right.$,
则$k=$ \fillin{$\dfrac{1}{2}$}。
\end{problem}

% \vfill

\begin{problem}
已知随机变量$\xi$的期望和方差各为$E\xi=3, D\xi=2$, 则$E\xi^2=$ \fillin{$11$}。
\end{problem}

% \vfill

\begin{problem}
电子管寿命$\xi$满足平均寿命为$1000$小时的指数分布,则它的寿命小于$2000$小时概率为 \fillin{$1-e^{-2}$}。
\end{problem}

% \vfill

\begin{problem}
已知$\xi$和$\eta$相互独立且$\xi\sim N(1,4), \eta\sim N(2,5)$,则$\xi-2\eta\sim$ \fillin{$N(-3,24)$}。
\end{problem}

% \vfill

% \newpage

\makepart{单选题}{共~8~小题,每小题~2~分,共~16~分}

\begin{problem}
下列各排列哪个是偶排列 \pickout{D}
\options{3712456}
	{36715284}
	{654321}
	{41253}
\end{problem}

% \vfill

\begin{problem}
若三阶行列式 $\left|\begin{array}{ccc}
  a_1 & a_2 & a_3\\
  2 b_1 - a_1 & 2 b_2 - a_2 & 2 b_3 - a_3\\
  c_1 & c_2 & c_3
\end{array}\right| = 2$,则 $\left|\begin{array}{ccc}
  a_1 & a_2 & a_3\\
  b_1 & b_2 & b_3\\
  c_1 & c_2 & c_3
\end{array}\right|=$ \pickout{A}
\options{1}
	{-1}
	{2}
	{-2}
\end{problem}

% \vfill

\begin{problem}
已知矩阵 $A = \left(\begin{array}{ccc}
  1 & 1 & 0\\
  1 & x & 0\\
  0 & 0 & 1
\end{array}\right)$ 其中两个特征值为 $\lambda_1 = 1$ 和 $\lambda_2
= 2$,则 $x=$ \pickout{B}
\options{2}
	{1}
	{0}
	{-1}
\end{problem}

% \vfill

\begin{problem}
二次型 $f = 4 x_1^2 - 2 x_1 x_2 + 6 x_2^2$ 对应的矩阵等于 \pickout{C}
\options{$\left(\begin{array}{cc}
  4 & - 2\\
  - 2 & 6
\end{array}\right)$}
	{$\left(\begin{array}{cc}
  2 & - 2\\
  - 2 & 3
\end{array}\right)$}
	{$\left(\begin{array}{cc}
  4 & - 1\\
  - 1 & 6
\end{array}\right)$}
	{$\left(\begin{array}{cc}
  2 & - 1\\
  - 1 & 3
\end{array}\right)$}
\end{problem}

% \vfill

\begin{problem}
对任何一个本校男学生,以$A$表示他是大一学生,$B$表示他是大二学生,则事件$A$和$B$是\pickout{B}
\options{对立事件}
	{互斥事件}
	{既是对立事件又是互斥事件}
	{不是对立事件也不是互斥事件}
\end{problem}

% \vfill

\begin{problem}
下列说法\uline{不正确}的是\pickout{B}
\options{大数定律说明了大量相互独立且同分布的随机变量的均值的稳定性}
	{大数定律说明大量相互独立且同分布的随机变量的均值近似于正态分布}
	{中心极限定理说明了大量相互独立且同分布的随机变量的和的稳定性}
	{中心极限定理说明大量相互独立且同分布的随机变量的和近似于正态分布}
\end{problem}

% \vfill

\begin{problem}
在数理统计中,对总体$X$和样本$(X_1,\cdots,X_n)$的说法哪个是\uline{不正确}的\pickout{D}
\options{总体是随机变量}
	{样本是$n$元随机变量}
	{$X_1, \cdots, X_n$相互独立}
	{$X_1 = X_2 =\cdots = X_n$}
\end{problem}

% \vfill

\begin{problem}
样本平均数$\bar{X}$\uline{未必是}总体期望值$\mu$的\pickout{A}
\options{最大似然估计}
	{有效估计}
	{一致估计}
	{无偏估计}
\end{problem}

\vfill

% \newpage

\makepart{计算题}{共~6~小题,每小题~8~分,共~48~分}

\begin{problem}
计算四阶行列式 $A = \left|\begin{array}{cccc}
  0 & 1 & 2 & 3\\
  1 & 2 & 3 & 0\\
  2 & 3 & 0 & 1\\
  3 & 0 & 1 & 2
\end{array}\right|$ 的值。
\end{problem}

\bigskip

\begin{solution}
$A = \left|\begin{array}{cccc}
    0 & 1 & 2 & 3\\
    1 & 2 & 3 & 0\\
    2 & 3 & 0 & 1\\
    3 & 0 & 1 & 2
  \end{array}\right| = \left|\begin{array}{cccc}
    0 & 1 & 2 & 3\\
    1 & 2 & 3 & 0\\
    0 & - 1 & - 6 & 1\\
    0 & - 6 & - 8 & 2
  \end{array}\right| = 1 \cdot (- 1)^{2 + 1} \left|\begin{array}{ccc}
    1 & 2 & 3\\
    - 1 & - 6 & 1\\
    - 6 & - 8 & 2
  \end{array}\right|$ \dotfill 4分\par
\qquad\qquad $= -\left|\begin{array}{ccc}
    1 & 2 & 3\\
    0 & - 4 & 4\\
    0 & 4 & 20
  \end{array}\right| = - \left|\begin{array}{cc}
    - 4 & 4\\
    4 & 20
  \end{array}\right| = -(-4\cdot20-4\cdot4) = 96$ \dotfill 8分
\end{solution}

% \vfill

\begin{problem}
用配方法将二次型 $f = x_1^2 + 2 x_1 x_2 - 6 x_1 x_3 + 2 x_2^2 - 12
x_2 x_3 + 9 x^2_3$ 化为标准形 $f = d_1 y^2_1 + d_2 y^2_2 + d_3 y^2_3$ 。
\end{problem}

\bigskip

\begin{solution}
$f = x_1^2 + 2 x_1 x_2 - 6 x_1 x_3 + 2 x_2^2 - 12 x_2 x_3 + 9 x^2_3$ \par
\qquad\qquad$= x_1^2 + 2 x_1 (x_2 - 3 x_3) + (x_2 - 3 x_3)^2 + x_2^2 - 6 x_2 x_3 $ \par
\qquad\qquad$= (x_1 + x_2 - 3 x_3)^2 + x_2^2 - 6 x_2 x_3$ \dotfill 3分 \par
\qquad\qquad$= (x_1 + x_2 - 3 x_3)^2 + x_2^2 - 2 x_2 \cdot 3 x_3 + (3 x_3)^2 - 9x_3^2$ \par
\qquad\qquad$= (x_1 + x_2 - 3 x_3)^2 + (x_2 - 3 x_3)^2 - 9 x_3^2$ \dotfill 6分\par
令$y_1 = x_1 + x_2 - 3 x_3, y_2 = x_2 - 3 x_3, y_3 = x_3$, \newline
则$f = y_1^2 + y_2^2 - 9y_3^2$为标准形。\dotfill 8分
\end{solution}

% \vfill

% \newpage

\begin{problem}
设二元随机变量$(\xi, \eta)$的联合分布表为
\begin{tabular}{|l|l|l|l|}
  \hline
  $\xi \backslash \eta$ & -1 & 0 & 1\\
  \hline
  0 & 0 & 1/3 & 0\\
  \hline
  1 & 1/3 & 0 & 1/3\\
  \hline
\end{tabular}。\par
(1) 求关于$\xi$和$\eta$的边缘分布。\par
(2) 判断$\xi$和$\eta$的独立性。\par
(3) 判断$\xi$和$\eta$的相关性。
\end{problem}

\bigskip

\begin{solution}
(1) 边缘分布为 \begin{tabular}{|l|l|l|}
  \hline
  $\xi$ & 0 & 1\\
  \hline
  $P$ & 1/3 & 2/3\\
  \hline
\end{tabular}, \ \begin{tabular}{|l|l|l|l|}
  \hline
  $\eta$ & -1 & 0 & 1\\
  \hline
  $P$ & 1/3 & 1/3 & 1/3\\
  \hline
\end{tabular}. \dotfill 2分 \par
(2) 由$P(\xi = 0, \eta = 0) = \frac{1}{3} \neq \frac{1}{9} = P(\xi = 0) P(\eta = 0)$,
知$\xi$和$\eta$不独立. \dotfill 4分 \par
(3) 由联合分布表求得$\xi \eta$的分布为 \begin{tabular}{|l|l|l|l|}
  \hline
  $\xi \eta$ & -1 & 0 & 1\\
  \hline
  $P$ & 1/3 & 1/3 & 1/3\\
  \hline
\end{tabular}.\dotfill 6分\par
因此有 $\cov(\xi, \eta) = E(\xi\eta) - E\xi E\eta = 0 -\frac{2}{3} \cdot 0 = 0$,
因此$\xi$和$\eta$不相关. \dotfill 8分
\end{solution}

% \vfill

\begin{problem}
设随机变量$\xi \sim N (1, 4)$,求$P (- 1 < \xi < 5)$。
\end{problem}

\bigskip

\begin{solution}
$P(-1<\xi<5) = \Phi_0\left(\frac{5-1}{2}\right) - \Phi_0\left(\frac{-1-1}{2}\right)$ \dotfill 2分 \par
\qquad $= \Phi_0 (2) - \Phi_0 (- 1)$ \dotfill 4分 \par
\qquad $= \Phi_0 (2) + \Phi_0 (1) - 1$ \dotfill 6分 \par
\qquad $= 0.9773 + 0.8413 - 1 = 0.8186$ \dotfill 8分
\end{solution}

% \vfill

% \newpage

\begin{problem}
设每发炮弹命中飞机的概率是0.2且相互独立,现在发射100发炮弹。\par
(1) 用切贝谢夫不等式估计命中数目$\xi$在10发到30发之间的概率。\par
(2) 用中心极限定理估计命中数目$\xi$在10发到30发之间的概率。
\end{problem}

\bigskip

\begin{solution}
$E\xi = n p = 100 \cdot 0.2 = 20, D\xi = n p q = 100 \cdot 0.2 \cdot 0.8 = 16$. \dotfill 2分 \par
(1) $P (10 < \xi < 30) = P (| \xi - E \xi | < 10) \geq 1 - \frac{D\xi}{10^2}
     = 1 - \frac{16}{100} = 0.84$. \dotfill 4分 \par
(2) $P (10 < \xi < 30) \approx \Phi_0 \left( \frac{30 - 20}{\sqrt{16}}\right)
     - \Phi_0 \left( \frac{10 - 20}{\sqrt{16}} \right)$ \dotfill 6分\par
\qquad $= 2 \Phi_0 (2.5) - 1 = 2 \cdot 0.9938 - 1 =0.9876$ \dotfill 8分
\end{solution}

% \vfill

\begin{problem}
从正态总体$N(\mu,\sigma^2)$中抽出样本容量为16的样本,算得其平均数为3160,标准差为100。
试检验假设$H_0:\mu=3140$是否成立($\alpha = 0.01$)。
\end{problem}

\bigskip

\begin{solution}
(1) 待检假设 $H_0 : \mu = 3140$. \dotfill 1分\par
(2) 选取统计量 $T = \frac{\bar{X}-\mu}{S / \sqrt{n}} \sim t(n-1)$. \dotfill 3分 \par
(3) 查表得到 $t_{\alpha} = t_{\alpha} (n - 1) = t_{0.01} (15) =2.947$. \dotfill 5分 \par
(4) 计算统计值 $t = \frac{\bar{x} - \mu_0}{s/\sqrt{n}} =\frac{3160-3140}{100/4} = 0.8$.\dotfill 7分 \par
(5) 由于 $| t | < t_{\alpha}$, 故接受 $H_0$, 即假设成立. \dotfill 8分
\end{solution}

% \vfill

% \newpage

\makepart{证明题}{共~2~小题,每小题~10~分,共~20~分}

\begin{problem}
不使用矩阵可相似对角化的判别定理,直接用矩阵的运算和性质证明下面的矩阵$A
=\left(\begin{array}{cc}
  1 & 1\\
  0 & 1
\end{array}\right)$不能相似对角化,即不存在可逆矩阵$P$和对角阵$\Lambda$使得$P^{-1}AP=\Lambda$。
\end{problem}

\bigskip

\begin{proof}
假设有$P = \left(\begin{array}{cc}
  a & b\\
  c & d
\end{array}\right)$使得$P^{-1}AP = \Lambda$,即$AP=P\Lambda$。\dotfill 2分\par
则有 $$\left(\begin{array}{cc}
  a + c & b + d\\
  c & d
\end{array}\right) = \left(\begin{array}{cc}
  1 & 1\\
  0 & 1
\end{array}\right) \left(\begin{array}{cc}
  a & b\\
  c & d
\end{array}\right) = \left(\begin{array}{cc}
  a & b\\
  c & d
\end{array}\right) \left(\begin{array}{cc}
  \lambda_1 & \\
  & \lambda_2
\end{array}\right) = \left(\begin{array}{cc}
  a \lambda_1 & b \lambda_2\\
  c \lambda_1 & d \lambda_2
\end{array}\right)$$ 因此有 $\left\{ \begin{array}{llll}
  a + c & = & a \lambda_1 & (1)\\
  b + d & = & b \lambda_2 & (2)\\
  c & = & c \lambda_1 & (3)\\
  d & = & d \lambda_2 & (4)
\end{array} \right.$ \dotfill 6分\par
由第1个和第3个方程消去$\lambda_1$,可以得到 $c^2 = 0$ 即 $c=0$;
由第2个和第4个方程消去$\lambda_2$,可以得到 $d^2 = 0$ 即 $d=0$。
因此矩阵$P$不可逆,矛盾。\dotfill 10分
\end{proof}

% \vfill

\begin{problem}
设事件$A$和$B$相互独立,证明$A$和$\bar{B}$相互独立。
\end{problem}

\bigskip

\begin{proof}
$P (A \cdot \bar{B}) = P (A - B) = P (A - A B)$ \dotfill 3分 \par
\qquad $= P (A) - P (A B) = P (A) - P (A) P (B)$ \dotfill 6分 \par
\qquad $= P (A) (1 - P (B)) = P (A) P (\bar{B})$ \dotfill 9分 \par
所以$A$和$\bar{B}$相互独立。\dotfill 10分
\end{proof}

% \vfill

\end{document}
