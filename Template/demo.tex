% !TEX encoding = UTF-8
% !TEX program = xelatex
\documentclass{USTBExam}

% =================================================
%       PDF信息
% =================================================
% PDF信息里的作者栏
\author{北京科技大学学生学习与发展指导中心·朋辈讲师团·高数组·黄腾}
% PDF信息里的主题
\Subject{版权所有,未经允许,不允许任何组织和部门以任何形式盗用}
% PDF信息里的关键词
\Keywords{高数; 微积分; 期中}

% =================================================
%       试卷头信息
% =================================================
% 试卷头里的年份
\Year{2020}
% 试卷头里的学期
\Semester{一}
% 试卷头里的课程
\Course{微积分AI}
% 是否期中
\Suffix{}
% 试卷头里的类型,如A/B/模拟等
\Type{A}
% 试卷头计分表中大题的数目
\TotalPart{4}

\begin{document}

% 生成试卷表头
\maketitle

\section{单项选择题(本题共10小题,每题4分,满分40分)}

\begin{problem}

\paren[]
\begin{choices}
	\item
	\item
	\item
	\item
\end{choices}
\end{problem}

\begin{problem}

\paren[]
\begin{choices}
	\item
	\item
	\item
	\item
\end{choices}
\end{problem}

\begin{problem}

\paren[]
\begin{choices}
	\item
	\item
	\item
	\item
\end{choices}
\end{problem}

\begin{problem}

\paren[]
\begin{choices}
	\item
	\item
	\item
	\item
\end{choices}
\end{problem}

\begin{problem}

\paren[]
\begin{choices}
	\item
	\item
	\item
	\item
\end{choices}
\end{problem}

\begin{problem}

\paren[]
\begin{choices}
	\item
	\item
	\item
	\item
\end{choices}
\end{problem}

\section{填空题(本题共10小题,每题4分,满分40分)}

\begin{problem}

\fillin{}.
\end{problem}

\begin{problem}

\fillin{}.
\end{problem}

\begin{problem}

\fillin{}.
\end{problem}

\begin{problem}

\fillin{}.
\end{problem}

\begin{problem}

\fillin{}.
\end{problem}

\begin{problem}

\fillin{}.
\end{problem}

\section{计算题(本题共2小题,每题6分,满分12分)}

\begin{problem}

\end{problem}

\begin{solution}

\end{solution}

\begin{problem}

\end{problem}

\begin{solution}

\end{solution}

\begin{problem}

\end{problem}

\begin{solution}

\end{solution}

\begin{problem}

\end{problem}

\begin{solution}

\end{solution}

\section{证明题(本题满分8分)}

\begin{problem}

\end{problem}

\begin{proof}

\end{proof}

\begin{problem}

\end{problem}

\begin{proof}

\end{proof}

\end{document}
